\documentclass[11pt]{amsart}
\usepackage{geometry}                % See geometry.pdf to learn the layout options. There are lots.
\geometry{letterpaper}                   % ... or a4paper or a5paper or ... 
%\geometry{landscape}                % Activate for for rotated page geometry
%\usepackage[parfill]{parskip}    % Activate to begin paragraphs with an empty line rather than an indent
\usepackage{graphicx}
\usepackage{amssymb}
\usepackage{epstopdf}
\usepackage{xcolor}
\usepackage{listings}
\lstset
{
    language=[LaTeX]TeX,
    breaklines=true,
    basicstyle=\tt\scriptsize,
    keywordstyle=\color{blue},
    identifierstyle=\color{magenta},
}


\DeclareGraphicsRule{.tif}{png}{.png}{`convert #1 `dirname #1`/`basename #1 .tif`.png}
\newcommand{\tab}{\hspace*{2em}}
\title{How to use the this {\LaTeX} template and Endnote to write your thesis}
%\date{}                                           % Activate to display a given date or no date

\begin{document}
\maketitle
%\section{}
%\subsection{}

% \vspace{5 mm}
\begin{enumerate}
\item Make sure you have all the requisite files
\begin{itemize}
\item \textit{chap1.tex}\\
This is a sample chapter file. Simply edit as you see fit and duplicate as you see fit. 
\item \textit{abstract.tex}\\
This is the abstract. 
\item \textit{main.bib}\\
This file gets generated from your bibliography.
\item \textit{main.tex}\\
This file contains the order of all the documents as they should appear in the proper order. 
\item \textit{apa-good.bst}\\
This contains the edited APA style template that does not contain any of the \textit{note} section found in Endnote when you search for references on Pubmed. The note section tends to include funding mechanisms, abstract, title and a bunch of other ancillary things. 
\item \textit{appa.tex}\\
This is the archetypical appendix
\item \textit{biblio.tex}\\
This is the standard bibliography, however, everything gets populated automatically from bibtex'ing the .bib file exported from endnote.
\item \textit{cover.tex}\\
This is the cover section that you should edit with your information as well as acknowledgements
\item \textit{cuthesis.cls}\\
This is the custom modified MIT class for Columbia, should not need adjustment.
\item \textit{lgrind.sty}\\
This shouldn't need adjustment. 
\item \textit{signature.tex}\\
This contains the signature portion. 
\item \textit{images} folder\\
This folder should be empty. It is where the images and figures go.
\end{itemize}
\item Now that you have all the items, the first thing to do would be to edit your cover page and abstract. Fill in the appropriate data. 
\item Next, take all of the references you have acquired in Endnote and edit each one such that the 'label' field contains a unique identifier. The standard is 'lastnamefirstauthoryear' so 'palazzo2015'. The actual name isn't important, and can be custom, however, it must be unique and memorable.
\item In Endnote, once the references have been updated, go to 'Export' and select 'Bibtex'. Save this somewhere memorable. Now, Endnote is piece of crap, and EXPLICITLY refuses to let you save this with the .bib extension. As such, you will need to save it in the directory where all of the other documents are and save it as 'main.bib.txt' and MANUALLY copy or move 'main.bib.txt' 'main.bib'. 
\item Citing is now a fairly straightforward process. To cite a work, simply do

\begin{lstlisting}
Energy and mass are like the same thing or whatever.\cite{einstein1908}
\end{lstlisting}

where 'einstein1908' is the unique identifier in your endnote bibliography.
\item To insert a picture, save the picture as a suitable file format (any of the major players will do, including pdf or eps) and name it something memorable like you did with the labels in Endnote. 'graph1.pdf' or 'image5.jpg' are fine. Then, to insert the image, simply use the following basic format
\begin{lstlisting}
\begin {figure}[htbp]
\centering
\includegraphics[width=\textwidth]{airplane}
\caption{On the topic of not being called 'Shirley'.\cite{nielson1980}}
\label{fig1}
\centering
\end{figure}
\end{lstlisting}
where there is a file with the word 'airplane' in the beginning in the images folder.
\item Edit and add as many sections as you wish. However, if you need more or fewer chapters, simply edit that in the 'main.tex' document, since that is what is included and typeset in the final pdf. 
\item When you've written a little and you want to see it formatted, perform the following procedure:
\begin{itemize}
\item Run latex on main.tex
\item Run bibtex on main.bib
\item Run latex on main.tex
\item Run latex on main.tex, again
\end{itemize}
This should get everything all nice and pretty. 
\item If anything is missing in the pdf, you can search the .tex files for specific text strings. 
\end{enumerate}


\end{document}  
